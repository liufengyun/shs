\section{Examination of Existing Definitions}

Free will is an eternal topic in philosophy. Throughout history, philosophers have proposed many different definitions of free will. Here let’s examine the most influential definitions as they must be the best ones that the intelligence of human beings could have achieved.

\subsection{Ability to Choose on the Basis of One's Desires}

Philosophers like Thomas Hobbes and David Hume think that free will is \emph{the ability to select a course of action as a means of fulfilling some desire}. Hobbes put it in \emph{Leviathan}(Chapter XXI):

\begin{quote}
A FREE-MAN, is he, that in those things, which by his strength and wit he is able to do, is not hindred to doe that he has a will to ... from the use of the word Free-will, no Liberty can be inferred of the will, desire or inclination, but the Liberty of the man; which consisteth in this, that he finds no stop in doing what he has the will, desire or inclination to doe.
\end{quote}

David Hume wrote in \emph{An Enquiry Concerning Human Understanding}(part 1):

\begin{quote}
By liberty, then, we can only mean a power of acting or not acting, according to the determinations of the will; this is, if we choose to remain at rest, we may; if we choose to move, we also may.
\end{quote}

There are two problems with this definition. First, it depends on the mental concept desire which we want to avoid according to the second condition of the criteria. Second, with the definition, we would be able to attribute free will to ants and chickens, which is in conflict with our  intuitions. We need a way to distinguish human beings from ants and chickens in terms of free will. Chickens and ants can choose based on their desires, but it seems they can’t deliberate on the choices.

\subsection{Ability to Choose on the Basis of One’s Desires and Values}

Timothy(2010) suggests an improvement to the previous definition that free will is \emph{the ability to deliberately choose on the basis of one’s desires and values}. Obviously, this definition is closer to our intuitions, and now we would be unable to attribute free will to ants and chickens.

However, in respect to the second condition of the criteria, this definition is worse, as it now additionally depends on some mental process with the usage of the word \emph{deliberatively}. And the word values here should be understood as a kind of \emph{value} beliefs, thus it further complicates the definition.

Another problem with the definition is that we can’t easily verify in a specific scenario whether an action is deliberate or not from a third-person perspective. However, this is not a difficulty in principle, as the agent can report the deliberation openly.

Timothy(2010) pointed out that a worry with this definition is that there are agents who deliberately choose to act as they do but who are motivated to do so by a compulsive, controlling sort of desire. For example, we can imagine following scenario:

\begin{quote}
There are two groups A and B who have feuds between them. And the children raised up in each group are instilled with hatred of the other group. Suppose one day a child X of group A encountered a child Y of group B, who is yelling for help in the water. X, being a good swimmer, thinks that Y is a member of the enemy group, so he chosed to be deaf to the cry and Y drowned.
\end{quote}

It’s obvious that the desires and beliefs of X are determined by his upbringing. Can we think that X is free? Different from Timothy(2010), I think yes, X is still free. However, to what extent should we blame X is another thing. Being raised up with bad values is like acting with false information or with important information missing. We’re still free even if we did something wrong due to false information or due to missing information, though the extent to which we should be blamed will be different. This is consistent with our intuitions.

\subsection{The Ability to Do Otherwise}

According to Timothy(2010), some philosophers succinctly characterize free will as \emph{the ability to do otherwise}. However, this definition suffers from multiple interpretations.

Compatiblists tend to provide a conditional interpretation. The general idea of the conditional interpretation is that I would do otherwise if it were the case that … , where the ellipsis is filled by a different fact than what was actually the case. This interpretation can’t prevent us from attributing free will to chickens and ants, thus conflicts with our intuitions.


However, incompatiblists think that \emph{the ability to do otherwise} requires something stronger -- the \emph{actual possibilities} that were open to me without supposing a different condition in the environment. This interpretation is still somewhat obscure, what do actual possibilities mean? If we’re destined to prefer some choice over others, then there are no actual possibilities at all. So actual possibilities can only be chosen without preference. However, we know the thought experiment that a rational horse would starve to death if there are two grasses in front of him with equal distance. Even if we do randomly choose one, then can we say that a random choice is free? In such cases, we will be unable to differentiate a deliberative random choice from a lucky(non-deliberative) random choice.

Another problem with the definition is that it’s difficult to check whether an agent has the ability to do otherwise or not in a specific scenario, as least the definition does not state explicitly how to perform such checks.

\subsection{Frankfurt’s Second-Order Desire}

Frankfurt makes two distinctions of desires. First, he distinguish \emph{effective} desires from \emph{non-effective} desires. An effective desire is one that actually issues in action. For example, if you wants to visit Paris, if you do go to visit Paris, then that’s an effective desire. Frankfurt identifies a person’s \emph{will} with their \emph{effective desires}.

Second, he distinguish \emph{first-order desires} from \emph{second-order desires}. A first-order desire is a desire for anything other than a desire; a second-order desire is a desire for a desire. For example, you might have a first-order desire to contact your ex-girlfriend, and also a second-order desire that you desire not to contact your ex-girlfriend. When a person does want the first order desire to be effective, when they want it to be their will, Frankfurt calls this a \emph{second-order volition}.

Then he distinguish two concepts \emph{freedom of action} and \emph{freedom of  will}. He thinks that a person’s actions are free in so far as they stem from their desires; that is, if they had desired differently, they would have acted differently. This definition is very close to that of Thomas Hobbes and David Hume, but for Frankfurt this is merely freedom of action. Freedom of will requires something more. He proposes that \emph{a person has free will just in case they have second-order volitions, and they can bring their first order desires into line with them}.

Frankfurt thinks that non-human animals and children don’t have second-order volitions, thus they only have free actions, but don’t have free will. Normal persons have second-order volitions, and they can control their desires, thus they have free will. A drug addict does not have free will, because his second-order volitions can’t control his effective desires.

One obvious drawback of Frankfurt’s definition is that it depends on the mental concept desire which we want to avoid as discussed in the second criterion.

Another worry with Frankfurt’s definition is, why stop with second-order? Why not third-order, fourth-order, or whatever? Can we say that one’s free if his highest-order desire is effective?

What if a person’s desires at all levels are implanted? Is the person still free? Frankfurt would say yes, but it’s disputable. And it seems unlikely to settle the dispute unless we arrive at an indisputable concept of desire. So I think that Frankfurt’s definition, though insightful, is far from satisfactory.
