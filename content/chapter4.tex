\section{A New Perspective of Responsibility}

Each human being must have been familiar with reward and punishment, this game has been played from our childhood, and will not stop until our death. It’s well known that in almost all countries, the laws usually don’t punish children until they reach a lawful age. However, interestingly, sometimes we punish a pet or reward a police dog. You might ask the question, what are the qualifications for an agent to be able to be rewarded or punished? And when it’s meaningful to reward or punish, when it’s meaningless to do so?

The concept responsibility is closely linked to reward and punishment. When we say someone is responsible for his actions, what we mean usually is that the man should be rewarded or punished for his actions. The same question above rephrased in terms of responsibility is, when it’s meaning to attribute responsibility to an agent, when it’s not?

\subsection{Traditional Views of Responsibility}

Generally there are two broad philosophical interpretations of the concept responsibility, namely \emph{merit-based view} and \emph{consequentialist view}\cite{sep-moral-responsibility}.

According to the \emph{merit-based view}, praise or blame would be an appropriate reaction toward the agent if and only if he merits such a reaction. For blame or praise to be appropriate, there must be actual alternatives open to the agent and the agent should have acted freely, as the slogan goes \emph{no responsibility without freedom}.

However, according to \emph{consequentialist view}, praise or blame would be appropriate if and only if a reaction of this sort would likely lead to a better change, e.g. the agent improves his behavior. For most compatibilists, blaming and praising are just means of social regulation.

\subsection{The New Perspective}

The interesting point is, with the new conception of free will as creativity, both \emph{merit-based view} and \emph{consequentialist view} apply.

\subsection{Thought Experiment}

We can put the new definition into test by analyzing a well-known thought experiment by Fischer (Responsibility \& Control, p176):

\begin{quote}
Black is a nefarious neurosurgeon. In performing an operation on Jones to remove a brain tumor, Black inserts a mechanism into Jones’ brain which enables Black to monitor and control Jones’ activities. Jones, meanwhile, knows nothing of this. Black exercises this control through a computer which he has programmed so that, among other things, it monitors Jones’s voting behavior. If Jones shows an inclination to decide to vote for Carter, then the computer, through the mechanism in Jones’s brain, intervenes to assure that he actually decides to vote for Reagan and does so vote. But if Jones decides on his own to vote for Reagan, the computer does nothing but continue to monitor without affecting the goings-on in Jones’ head. Suppose Jones decides to vote for Reagan on his own, just as he would have if Black had not inserted the mechanism into his head.
\end{quote}

In the experiment above, can we think that Jones exhibited free will in voting? I think yes. Indeed, free will can be manipulated in three possible ways:

\begin{itemize}
\item influence what action plans are generated
\item change the objective
\item make the decision less rational
\end{itemize}

A typical example is shopping. The seller would try to manipulate your free will through all the three ways. For example, she may falsely inform you that a new technology has outdated old products that lack this technology, so you should not consider some choices. Then she would make you believe that some quality features are more important than others, thus change your objective. Finally, the seller may make you feel difficult to reject her recommendation as she looks so nice and beautiful.

In the example above, we still think you are free in the shopping, despite the false information and the rash decision. You are free because you do generate plans and make almost rational choice.

Now turn to the neurosurgeon experiment. In principle, I think, to tamper with one’s neural system is not very different from using one of the three ways above to influence one’s decision. If we’re willing to attribute free will to the person in the shopping example, then I think it’s reasonable to think that Jones exhibited free will in voting.

In the next two sections, we will revisit the problem of determinism and responsibility from the new definition of free will.

\subsection{Free Will and Responsibility}

I tend to side with the consequentialist view here. I think the purpose of reward and punishment is to improve the free will. If reward or punishment can’t improve free will, then it does not make sense to do so. For example, it does not make sense to reward a chess program or punish a car.

How can reward or punishment improve free will? For human beings, I think there are three ways that free will can be improved through reward or punishment:

\begin{itemize}
\item influence generation of action plans
\item adjust objectives
\item decide more rationally
\end{itemize}

For example, reward may lead to generation of more action plans in a particular direction, while punishment generate less plans in another direction. Reward or punishment may make one objective more important and stronger than others. Also, reward and punishment can make one to decide more rationally just as one can improve the ability to do elementary arithmetics in primary school through reward and punishment.

Note that reward or punishment are not the only possible way to improve free will. At least for human beings, learning and reflection can also improve our free will. I think this insight  can help us to reflect on blame or punishment in child education or company management.
