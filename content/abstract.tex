\begin{abstract}

Free will has been studied for more than two thousand years. A dominant view among philosophers is to regard free will as the capability to choose, as reflected by the popular formulation of free will as \emph{the ability to do otherwise}. I label this view of free will as \emph{the myth of choice}. In this paper, I'll argue that \emph{the myth of choice} is flawed, and it's not the capability to choose but the capability to generate choices that matters.

In the real world, agents are first faced with problems, then they generate candidate choices for decisions and actions. More \emph{creative} an agent is, more and better choices can be generated, decided and acted upon, thus more free the agent is. Free will is first and foremost related to \emph{creativity}. Free will is \emph{creative}, it varies in degrees and can be improved.

% The new conception is closer to our experience, and answers a lot of questions that traditional philosophers feel difficult to answer, such as when a child has free will. The new conception also provides a new perspective on responsibility, the insights will no doubt be very useful in education and management regarding awarding and punishment.

% Past philosophers either think \emph{free will} as too noble and commit too much to it -- for example, assume the existence of \emph{causally efficacious} and \emph{non-physical} agents, which contradicts the contemporary scientific belief about the \emph{causal closedness} of the world. Or, they think \emph{free will} as too humble, such as admitting there's no \emph{real alternatives}, which makes it difficult to support ideas like human dignity, choices and moral responsibility, which are thought to be the foundation of human society.

% The new conception gives us more things that the second approach can't provide. For example, human beings can still be proud of their free will, as human creativity is orders higher than other creatures in a lot of aspects. We still have choices as long as we can learn to be more creative, or more rational, or update our values. We can still be held as responsible if our values can be updated, or our creativity or rationality can be improved.

\end{abstract}
