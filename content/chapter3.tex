\section{Free Will Is Creative}

The philosopher Jean-Paul Sartre has a famous quote saying that \emph{man is condemned to be free}. For Sartre and many other philosophers, free will seems to be a magic power, once owned by an agent, can be exerted on any matters of affair. According to this view, free will is an inexhaustible static asset for agents. I'll argue in this section that free will is not a static asset, but a dynamic asset. Free will is not universal money that can buy things everywhere, but coupons that must be used at the corresponding stores.

\subsection{The Myth of Choice}

The dominant view among philosophers is to regard free will as some capacity to choose, as reflected by the popular formulation of free will as \emph{the capacity to do otherwise}. We can see the traces of this view in all debates about free will. It's the view advocated by libertarians, but also the view that compatibilists are struggling with. I call this view \emph{the myth of choice}. If we were able to break this myth, then we would fundamentally change how people will debate and talk about free will in future. This is the task I'm going to undertake here.

In the beginning of this paper, we've seen an example where you can decide freely on whether to give the coin in your pocket to the street artist or not. It seems that each choice is at your disposal, it's completely to you to decide. For both traditional philosophers and ordinary people, this feel of ability to make free choice seems to be essential to the concept of free will.

First, I'd like to say, \emph{free will is not hesitation}. When you can't decide between two choices, you don't know what to do next, you feel unease about the uncertainty, you want to get out of the situation. You linger on one choice, while looking at the other. You intend to decide on one, but you hesitate. You swing like a pendulum between the two choices. This is a trouble instead of freedom!

Some might question: doesn't the ability to hesitate exhibit free will? If hesitation reflects free will, then we can write a simple program to exhibit free will as well:

\begin{lstlisting}[language=Scala]
  def hesitate(opt1, opt2) = {
    if (opt1 betterThan opt2)
       choose(opt1)
    else
       hesitate(opt2, opt1)
  }
\end{lstlisting}

In the code snippet above, we define a function \emph{hesitate}, which takes two choices \emph{opt1} and \emph{opt2} as parameter. If the first option is better than the second option, it chooses the first. Otherwise, it recursively calls itself by reversing the parameters. If we feed two options of which one is better, the program will choose the better one and terminate. However, if we feed two options that are equally better, the program never terminates -- it hesitates between the two options!

If we think it doesn't make sense to attribute free will to the program above, then it also doesn't make sense to think that hesitation exhibits free will.

Second, \emph{free will is not randomness}. When facing two equally good choices, we often resort to the strategy of random choice. Resorting to random choice doesn't necessarily mean resort to a true random process in the physical world. It just means to give up our right to decide, and let external forces -- which we can't predict the result -- to decide. As we are passive and determined by external forces when we resort to randomness in decision, so the process doesn't reflect free will at all.

One may disagree on one point: isn't the act of resorting to randomness in case of equally good choices an exhibition of free will? Maybe, but the ability can also be implemented by a simple program:

\begin{lstlisting}[language=Scala]
  def decide(opt1, opt2) = {
    if (opt1 betterThan opt2)
       choose(opt1)
    else if (opt2 betterThan opt1)
       choose(opt2)
    else
       choose(random(opt1, opt2))
  }
\end{lstlisting}

In the code snippet above, we define a function \emph{decide}, which chooses the best one if the two choices are not equally good. Otherwise, it calls a random process to randomly choose one. Note that the \emph{random} method doesn't have to be real random, it could be just pseudo-random.

Is it possible for any conception of free will be deviant from the program regarding the capacity to choose? I can hardly imagine any. First, the decision process should be rational to some extent, i.e. to choose the better choice, or it doesn't qualify the name of free will. We never attribute free will to a man who decides irrationally. Instead in such cases we doubt if the person has some psychological disorder. The rational decision process will generate a unique result as long as there's only one best answer. Second, in case there are more than one best answers, the decision process should choose one randomly, to avoid the paradox of Buridan's ass\footnote{\url{http://en.wikipedia.org/wiki/Buridan's_ass}}, where a rational horse starves to death in front of two stacks of hay.

As I've showed above, there's nothing interesting about the capacity to choose. It's this myth of choice that misled philosophers for more than thousand years. It's time to give up this myth of free will.

\subsection{It's About Choices, Not Choice}

To break a myth is not enough, we have to replace it with another one. Next I'm going to argue that it's not the capability to choose but the capability to generate choices that matters.

Nearly all definitions of free will begin with the situation where the agent is about to make a choice. Some philosophers emphasize the ability to make a different choice, some emphasize the alignment between choice and action.

But making a choice is not the primitive situation we face in daily lives. In the real world, we first face a problem, then we generate candidate choices for decisions. We've seen the capacity to decide is not interesting at all, and we hesitate to say it exhibits free will. Then the only possibility lies with the choices. The more creative we are, the better choices can be generated and decided, thus more free we are. Free will is first and foremost a problem related to \emph{creativity}. It's about choices, not choice!

From a linguistic point of view, what past philosophers care about is the adverb which modifies the verb in following sentence:

\begin{quote}
  The agent \emph{freely} decided on one among the three options.
\end{quote}

Instead, the approach I advocate in this paper would rewrite the sentence and emphasize the \emph{options} part:

\begin{quote}
  The agent decided on the best one among \emph{the three options he created}.
\end{quote}

Note that in the rewriting, \emph{freely} is removed while \emph{best} is added. This corresponds to the conclusion in previous section that the capacity to choose doesn't quality the name of free will and the decision process should be rational to choose the best option.

If the three options are indeed created by the agent himself, the rewriting literally means the same thing. If the three options are provided by third parties, we need more complex rewriting. When we face choices provided by third parties, it's first and foremost a problem(not choice), and second a problem related to choice. We can generate different choices in order to solve the choice problem. We have to decide on one choice we generated in order to decide on the choice problem.

For example, John receives two offers from two universities, he has to decide which one to decline. He would generate two choices in order to decide on the choice problem:

\begin{itemize}
\item Choose the university which has the top overall ranking
\item Choose the university which has the top specialty ranking
\end{itemize}

After generating the two choices, John decided on the second one, because he's persuaded by his father that the ranking of specialty is more important than school. Then, he would be able to choose a university by the criteria he has chosen.

It's obvious that the quality of the decision mainly depends on the quality of the choices that are generated. The better choices we can generate, the greater chance we can make a better decision. If we move the focus of free will from the capacity to choose to the capacity to generate choices, we see a very different scenery. It would cast new light in the understanding of free will and reshape the debates centered around free will.

To make the discussion more concrete, let me first put forward a new definition of free will formally:

\begin{quote}
\emph{Free will} is the ability to generate action plans and choose one rationally with respect to a specific objective in a concrete scenario.

% \emph{Free action} is execution of an action plan chosen by free will.
\end{quote}

This definition doesn't sound as good as \emph{free will is creative}, but in spirit they are the same. In the following, I'll analyze this new definition in details.

% The definition follows Frankfurt to distinguish between \emph{free will} and \emph{free action}. The distinction is justified by the fact that sometimes we are able to will freely but unable to fulfill the will due to obstacles in the physical world.

\subsection{Creativity}

The definition depends on the concept \emph{action plan}, which is a list of actions to be executed in order to achieve a specific objective in a concrete scenario. When facing a problem, each action plan is a choice. Action plan is relative to the objective, each action in the action plan can also be an objective for sub-action plans. For example, suppose visit Paris is an action plan for having a good holiday. In turn, go to Paris by train is a possible step of an action plan for visiting Paris.

The interplay between action plan and objective can be very complex. When we are faced with a problem, to solve this problem is not what I mean by \emph{objective}. Objective means what the agent can gain from solving the problem. Take the previous example of going to university, what's the objective? I can imagine a few:

\begin{itemize}
\item Receive a good education in the field of interest
\item Receive a good general education
\item Have a good job in future
\item Have a good life in future(not only professional, but overall)
\end{itemize}

As you can see, the objectives can be of different levels and different aspects. Then how do we fix on one objective in order to decide? This is actually a very creative process. First, it's reasonable to assume that each agent has some values, or it's impossible for the agent to make any reasonable decisions. The creative process is try to create a network of bridge the original problem with the values, as illustrated in the following graph:

\begin{center}
  \tikzstyle{line} = [draw, -latex']

  \begin{tikzpicture}[every node/.style={draw, font=\small, text centered, rounded corners, node distance=3cm}, choice/.style={circle, draw}]
    \node  (problem1) {Problem 1};
    \node [choice, fill=red, above right of=problem1] (1a) {1};
    \node [choice, right of=problem1] (1b) {2};
    \node [choice, below right of=problem1] (1c) {3};
    \node [right of=1b] (problem2) {Problem 2};
    \node [choice, above right of=problem2] (2a) {1};
    \node [choice, fill=red, below right of=problem2] (2c) {3};
    \node [below right of=2a] (problem3) {Problem 3};
    \node [choice, right of=problem3] (o3) {Values};

    \path [line] (problem1) -- (1a);
    \path [line] (problem1) -- (1b);
    \path [line] (problem1) -- (1c);

    \path [line] (1a) -- (problem2);
    \path [line] (1b) -- (problem2);
    \path [line] (1c) -- (problem2);

    \path [line] (problem2) -- (2a);
    \path [line] (problem2) -- (2c);

    \path [line] (2a) -- (problem3);
    \path [line] (2c) -- (problem3);

    \path [line] (problem3) -- (o3);
  \end{tikzpicture}
\end{center}

In the graph above, in order to solve the first problem, we generate three options, which in turn becomes a problem itself. To solve the second problem, we generate two options. Now the third problem can be solved using the values of the agent as objective function, the option 3 is chosen. In turn, the option 3 can serve as objective function for problem 2, so the option 1 is chosen -- that's how the first problem is solved. In reality, the graph construction process may begin at both ends in order to bridge the original problem with values of the agent. The intermediate options generated depends on the agent's beliefs, skills, it's a very creative process.

Human beings can profit their linguistic capabilities to generate action plans. The simplest method is to negate a sentence. For example, from \emph{visit Paris} we can get \emph{don’t visit Paris}. Substitution is also a widely used technique. For example, instead of going to visit Paris, one may go to Geneva. The natural language is a huge source of human creativity in generating action plans.

Note that the quality of action plans matters more than the quantity. The quality of an action plan is determined by how well it can achieve the specific objective in the concrete scenario. For example, if you want to go from Geneva to Paris in the shortest delay, there are a lot of transportation possibilities, among them one is the best. A person generating the best action plan is more creative than the person who can’t, even though the latter might generate more but lower quality plans. However, in many scenarios, quantity matters as well, as only through comparison we can be sure if a choice is optimal or not. Shopping is a typical example.

According to the new definition, free will is creative -- the better action plans are, more free the agent is. The creativity of agents varies in degrees. A person is more creative than a child or a dog in how to get the banana which they can’t reach. A chess program is usually more creative than a chess master in generating alternative plans, though they generate plans by completely different processes. A poet is more creative than a computer in writing poems.

The definition also implies that free will comes in degrees. In the same scenario with the same objective, the agent who can generate and decide on better action plan is more free than the other one.

This definition implies free will can be improved. If an agent is able to learn to generate better action plans and decide more rationally, it will become more free. It’s consistent with our intuitions, and can easily provide an answer to the question \emph{when does a child have free will}. A chess program that can learn from failures to generate better action plans in future will become more and more free. Does this mean free will is a synonym of intelligence? To some extent, I admit the two concepts are closely related. However, as intelligence is an obscure concept to be defined, I will not pursue the topic here.

\subsection{Thematicity}

Free will is not a universal currency that can be used everywhere, it's more like tickets which can be used at specific places.

Talk of free will should be restricted to a proper topic. The correct grammar for talking about free will is \emph{X exhibits free will in T}. The topic T can’t be omitted, or the sentence is incomplete and incomprehensible unless T is implicit in the context. For example, chess program exhibits free will in playing chess games, but not in speaking english. A child exhibits free will in movements of one’s body(mouth, hand, foot, etc), but not in the circulation of blood, digestion or cell fission. However, a doctor may exhibit free will in circulation of blood, digestion and even cell fission by employing his professional knowledge to make plans to tune the functioning of his body. A musician may  exhibit free will in music composition, but a man who doesn’t know music at all can’t.

Free will is not a nature of human beings, but a nurture. The more one learns on a topic, the more creative thus more free he would be on the topic. If one doesn't know anything about a topic, they are not very different from a cow attending a concert. Millionaires are never respected by artists just because they buy and sell masterpieces of art.

For human beings, it's impossible for one to be an expert on all the topics he faces in life. So we delegate health to doctors, social issues to politicians, education of child to school, etc. However, there is one thing we can't delegate, that's how to spend the whole life. If we just follow the prototypical life roads of the public opinion, we're not showing any free will on the topic of life, even though we are very successful in our professions. To be more free in one's life, we have to learn how to be more creative in life choices.

\subsection{Rationality}

As I argued before, the decision process of free will can't be irrational. However, there’s a weakness here because the rationality of different agents may vary in degrees, far from being perfect. Generally, if we can justify the decision of an agent with respect to the specific objective in the concrete scenario based on the history of the agent, then we can say that the agent decides rationally.

However, generate action plans and choose rationally don’t necessarily imply that free will involves deliberative processes. The process may also be consciousless computations. For example, a chess program can also generate action plans internally and choose the best one in a concrete scenario in order to win the chess.

\subsection{Summary}

In this section, I put forward a new definition of free will. This new conception of free will differs from existing definitions in that it emphasizes the choices instead of how the decision is made.

% I showed that the new definition is compatible with our experiences or explains our experience in places of seemingly conflict.

According to new conception of free will, human beings can still be proud of their free will, as human creativity is orders higher than other creatures in a lot of aspects. We still have choices as long as we can learn to be more creative.

% One question I don't touch in this chapter nor I'll in this paper is, whether the new conception of free will is compatible with respect to determinism or indeterminism. I'll safely leave the question for later investigations, as whether the new conception of free will is acceptable or not doesn't depend on the answer to this question. Nevertheless, I believe the new conception of free will would incur less metaphysical disputes, and it's very promising to be compatible with indeterminism.

In the next section, I'm going to discuss possible attacks on the new conception of free will and respond to them.

% In the next section, we'll see that the new conception opens a new perspective to our understanding of responsibility, and how we use reward and punishment in practice.

% we can still be held as responsible if our creativity or rationality can be improved. The
