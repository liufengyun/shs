\section{Introduction to the Problem of Free Will}

Imagine that you're walking along a street and there's an artist playing violin on the road side. When you're approaching the artist, you might ask yourself: \emph{should I give him the coin in my pocket?}

The question you asked yourself implies that you believe firmly that (1)there are alternatives how the world will proceed, (2)what's going to happen next solely depends on your decision, and (3)you can effectively create a course of intended events in the world to change how it proceeds. Usually we denote the intuition under the name \emph{free will}.

A popular characterization of free will in philosophy is \emph{the ability to do otherwise}. It presupposes there are real choices of the world, and we are able to make decisions and effectively act on the decisions to make changes in the world.

However, this conception of free will is in conflict with the scientific world view: the world is composed of basic particles, governed by physical laws. Under this basic world view, there’s no place for free will no matter the world is determined or not.

If we assume \emph{determinism}, i.e. any event is a necessary consequence of the physical laws and the past, then the present and future are completely determined by the past and the physical laws. It means all our desires, beliefs and actions are determined as well, it’s impossible for anyone to change the proceedings of the world, there’s no real choices at all. This argument is often labeled as \emph{consequence argument} in the context of determinism.

If we assume \emph{indeterminism}, i.e. events of the world are not the necessary consequence of the physical laws and the past, then at least some events can happen by chance. If our beliefs and desires originate from random processes, it’s impossible to distinguish free will from chance. This argument is often labeled as \emph{matter of chance argument} in the context of indeterminism.

The crisis of free will in metaphysics also endangers our fundamental belief in ethics. Imagine the situation where a witness lied in the court, it's reasonable for us to blame him. However, if the witness is compelled to do so by death threat to him or his family members, we would be reluctant to blame him for his false testimony in court. The difference lies in whether the man doing the deed is free or not. In another world, one is only responsible for his free actions. However, if free will is impossible both in a determined world and in an undetermined world, then it seems difficult to maintain a coherent view of \emph{responsibility}.

In face of the tension between free will and the scientific world view, philosophers divide mainly into two camps, namely \emph{libertarianism} and \emph{compatibilism}.

\subsection{Libertarianism}

Libertarianists propose that there exists \emph{causally efficacious} and \emph{non-physical} agents, which is the source of our free will. This school of thought is often labeled as \emph{libertarianism}.

\emph{Libertarianism} trivially avoids the \emph{consequence argument} and \emph{matter of chance argument}. \emph{Consequence argument} doesn't apply as libertarianism presupposes indeterminism. \emph{Matter of chance argument} becomes invalid because a causally efficacious but non-physical agent is capable of rational choices among genuine alternatives.

\emph{Libertarianism} does manage to solve the tension between free will and the scientific world view, however, by paying a price -- it contradicts the contemporary scientific world view that the physical world is \emph{complete} and \emph{causally closed}.

The price paid is not that high as its first look. As we know, the physicists' dictionary is expanding and changing over time. Things like electromagnetic fields, photons and various new particles were added in last century, while things like phlogiston and ether were removed. There does not exist a strict boundary between \emph{physical} and \emph{non-physical}. In the history of science, it seems that anything that helps explain the world can appear in the dictionary thus become physical. When Newton postulated the concept \emph{force} and gave it a definition, no physicists cared about its metaphysical implications as long as it \emph{worked}.

If \emph{libertarianism} has the best explanatory power and is the simplest among all candidate theories, I think the causally efficacious agents postulated by \emph{libertarianism} would be added to physicists' dictionary as well when the mainstream physicists are interested in the problem.

\iffalse
I think the problem of libertarianism doesn't lie in the conflict with physics, but in the conflict of \emph{real choices} and \emph{rationality}. The agents postulated by proponents of libertarianism are not only causally efficacious, but also rational, as they don't cause events randomly in the world. As we know from the paradox of Buridan's ass\footnote{\url{http://en.wikipedia.org/wiki/Buridan's_ass}}, as a rational horse would starve to death in front of two stacks of hay, a rational agent is unable to make a decision in case of real alternatives.

In the real world, we have a strategy in face of the problem: \emph{In case of two perfectly equal choices, make a random choice}. This strategy presupposes indeterminism, as random choice is impossible in a determined world. Can the agents of libertarianists follow the same strategy? If they do exactly the same, then it seems that we're able to produce a machine in an undetermined world  -- some events are random -- as follows:

\begin{quote}
Among the candidate choices, use a deterministic program to select the best ones accord to an objective function. If the selection has only one choice, then return it as result. Else, start a random process to select one from the best ones randomly, and return it as result.
\end{quote}

The machine does exactly the same as the agents of libertarianists, then why postulate additional entities? Whether we can cut the mustache of libertarianists using Occam's razor depends on whether the machine is possible or not in principle. I'll discuss the possibility of this machine in later this paper.
\fi

\subsection{Compatibilism}

On the other camp, philosophers suggest that free will is actually compatible with determinism. If we formulate the argument about the incompatibility between free will and determinism as follows:

\begin{quote}
  \begin{itemize}
    \item P1 If determinism is true, then every human action is causally necessitated
    \item P2 If every action is causally necessitated, no one could have acted otherwise
    \item P3 One only has free will if one could have acted otherwise
    \item P4 Determinism is true
    \item C No one has free will
  \end{itemize}
\end{quote}

Nearly all compatibilists deny P2 or P3, though they have nuances in their stances. The famous compatibilist, David Hume, wrote in \emph{An Enquiry Concerning Human Understanding}(part 1):

\begin{quote}
By liberty, then, we can only mean a power of acting or not acting, according to the determinations of the will; this is, if we choose to remain at rest, we may; if we choose to move, we also may.
\end{quote}

Hume intended to say that free will is the accord between desire and actions. It doesn't imply anything about the ability to do otherwise or existence of real choices. It's obviously this redefinition is compatible with determinism.

If we question Hume \emph{does a dog have free will}, he would probably answer yes. But this seems cheating, as it changes the definition of free will!

I have to argue here that redefinition of free will is a plausible approach. First, there doesn't exist an absolute authoritative definition of free will among philosophers. Some definitions are popular than others, but popularity doesn't mean truth. Second, the practical usage of the word \emph{free will} in natural languages can't serve as a criteria, as practical usage of words are abundant with inaccuracies,  errors and inconsistencies. Third, from an instrumentalist point of view, any definition would work as long as it is coherent and is able to explain our experiences.

Note that to some extent, the approach I'll take in this paper is similar to compatibilism, that's to redefine free will. However, regarding the definition itself, I take a very different approach from libertarianists and compatiblists, while the latter two are similar. I'll explain more about this in next chapter.

The general problem with various form of compatibilism is that it advocates a very humble definition of free will, in which human beliefs and desires are determined as well. This doesn't imply compatibilism is wrong, but we're more likely to buy a conception in which human dignity is preserved and human beings are not just determined and passive machines?

\subsection{Summary}

In this chapter, I briefly introduced the problem of free will, and two classical approaches to the problem of free will. The emergence of quantum physics and neural science in the last century led some scholars to rethink the possibility of free will in an undetermined world. The details of this new approach is abundant with technical details, so I omitted the approach here. Interested readers can refer to relevant materials in the field.

I label all the three approaches above as traditional approaches, as they -- though different -- share the same trait in how they define free will. Their definitions are all about how agents decide or act on choices. In the next section, I propose a new definition that focuses on the choices themselves, and I'll argue in detail why this is important for the conception of free will.
