\section{Introduction}

Imagine that you're walking along a street and there's an artist playing violin on the road side. When you're approaching the artist, you might ask yourself: \emph{should I give him the coin in my pocket?}

The question you asked yourself implies that you believe firmly that (1)there are alternatives how the world will proceed, (2)what's going to happen next depends on your decision, and (3)you can effectively create a course of intended events in the world to change how it proceeds. This is almost what we mean by \emph{free will} in daily talks.

A popular characterization of free will in philosophy is \emph{the ability to do otherwise}\cite{sep-freewill}. This formulation regards free will as the capability to choose. I label this view of free will as \emph{the myth of choice}. In this paper, I'll argue that \emph{the myth of choice} is flawed, and it's not the capability to choose but the capability to generate choices that matters.

In the real world, agents are first faced with problems, then they generate candidate choices for decisions and actions. More \emph{creative} an agent is, more and better choices can be generated, decided and acted upon, thus more free the agent is. Free will is first and foremost related to \emph{creativity}. Free will is \emph{creative}, it varies in degrees and can be improved.

Before we start our adventure for a new conception of free will, let's see why there're a lot of noises on the topic of free will for thousand of years.

\subsection{Problem of Free Will}

The popular view of free will as the ability to do otherwise is in conflict with the metaphysical views of determinism and indeterminism -- there’s no place for free will no matter the world is determined or not.

If we assume \emph{determinism}, i.e. any event is a necessary consequence of the physical laws and the past, then the present and future are completely determined by the past and the physical laws. It means all our desires, beliefs and actions are determined as well, it’s impossible for anyone to change the proceedings of the world, there’s no real choices at all. This argument is often labeled as the \emph{consequence argument} in the context of determinism.

If we assume \emph{indeterminism}, i.e. events of the world are not the necessary consequence of the physical laws and the past, then at least some events can happen by chance. If our beliefs and desires originate from random processes, it’s impossible to distinguish free will from chance. This argument is often labeled as the \emph{matter of chance argument} in the context of indeterminism.

The crisis of free will in metaphysics also endangers our fundamental belief in ethics. Imagine the situation where a witness lied in the court, it's reasonable for us to blame him. However, if the witness was compelled to do so by death threat to him or his family members, we would be reluctant to blame him for his false testimony in court. The difference lies in whether the man doing the deed is free or not. In another word, one is only responsible for his free actions. However, if free will is impossible both in a determined world and in an undetermined world, then it seems difficult to maintain a coherent view of \emph{responsibility}.

In face of the tension between free will and the scientific world view, philosophers divide mainly into two camps, namely \emph{libertarianism} and \emph{compatibilism}.

\subsection{Libertarianism}

Libertarians propose that there exists \emph{causally efficacious} and \emph{non-physical} agents, which is the source of our free will. This school of thought is often labeled as \emph{libertarianism}.

\emph{Libertarianism} trivially avoids the \emph{consequence argument} and \emph{matter of chance argument}. \emph{Consequence argument} doesn't apply as libertarianism presupposes indeterminism. \emph{Matter of chance argument} becomes invalid because a causally efficacious but non-physical agent is capable of rational choices among genuine alternatives.

\emph{Libertarianism} does manage to solve the tension between free will and the scientific world view, however, by paying a price -- it contradicts the contemporary scientific world view that the physical world is \emph{complete} and \emph{causally closed}.

The price paid is not that high as its first look. As we know, the physicists' dictionary is expanding and changing over time. Things like electromagnetic fields, photons and various new particles were added in last century, while things like phlogiston and ether were removed. There does not exist a strict boundary between \emph{physical} and \emph{non-physical}. In the history of science, it seems that anything that helps explain the world can appear in the dictionary thus become physical. When Newton postulated the concept \emph{force} and gave it a definition, no physicists care about its ontological status as long as it \emph{works} in predicting the movements of the heavenly bodies.

If \emph{libertarianism} has the best explanatory power and is the simplest among all candidate theories, I think the causally efficacious agents postulated by \emph{libertarians} would be added to physicists' dictionary as well when the mainstream physicists are interested in the problem.

\subsection{Compatibilism}

On the other camp, philosophers suggest that free will is actually compatible with determinism, this view is usually labeled as \emph{compatibilism}\cite{sep-compatibilism}. We can formulate the argument for the incompatibility between free will and determinism as follows:

\begin{quote}
P1 If determinism is true, then every human action is causally necessitated \\
P2 If every action is causally necessitated, no one could have acted otherwise \\
P3 One only has free will if one could have acted otherwise \\
P4 Determinism is true \\
C No one has free will
\end{quote}

Nearly all compatibilists deny P2 or P3, though they have nuances in their stances. The famous compatibilist, David Hume, wrote in \emph{An Enquiry Concerning Human Understanding}(part 1):

\begin{quote}
By liberty, then, we can only mean a power of acting or not acting, according to the determinations of the will; this is, if we choose to remain at rest, we may; if we choose to move, we also may.
\end{quote}

Hume intended to say that free will is the accord between desire and actions. It doesn't imply anything about the ability to do otherwise or existence of real choices. It's obviously this redefinition is compatible with determinism.

If we question Hume \emph{does a dog have free will}, he would probably answer yes. But this seems cheating, as it changes the definition of free will!

I have to argue here that redefinition of free will is a plausible approach. First, there doesn't exist an absolute authoritative definition of free will among philosophers. Some definitions are popular than others, but popularity doesn't mean superiority. Second, the practical usage of the word \emph{free will} in natural languages can't serve as a criteria, as practical usage of words are usually abundant with inaccuracies,  errors and inconsistencies. Third, from an instrumentalist point of view, any definition would work as long as it is coherent and is able to explain our experiences.

Note that to some extent, the approach I'll take in this paper is similar to compatibilism, that's to redefine free will. However, regarding the definition itself, I take a very different approach from libertarians and compatiblists, while the latter two are similar. The latter two approaches regard free will as related to the capacity to choose, but I think it's the capacity to generate choices that's important. I'll explain more about this in next chapter.

The general problem with all forms of compatibilism is that it advocates a very humble definition of free will, in which human beliefs and desires are determined as well, and it's impossible to differentiate human beings from dogs and cats. This doesn't imply compatibilism is wrong, but we're more likely to buy a conception in which human dignity is preserved and human beings are not just determined and passive machines.

\subsection{Summary}

In this chapter, I briefly introduced the problem of free will, and two classical approaches to the problem of free will. The emergence of quantum physics and neural science in the last century led some scholars to rethink the possibility of free will in an undetermined world. The details of this new approach is too technical, so I please interested readers to refer to relevant materials in the field.

The dominant view in the history of philosophy and which is still popular today is to regard free will as the capacity to choose. I label this view as \emph{the myth of choice}. I'll argue in next section that this view is flawed, and it's not the capability to choose but the capability to generate choices that matters. Based on this thought, I put forward a new conception of free will which is creative, comes in degrees and can be improved.
