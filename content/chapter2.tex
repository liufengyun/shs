\section{Possible Attacks and Responses}

This section I'll discuss possible attacks on the new conception and respond to them.

\subsection{Do Dogs Have Free Will}

The explicit requirement of generating action plans and choosing rationally prohibits us from attributing free will to trees and functional devices. Trees get rid of their leaves in autumn to protect their lives, but there’s no evidence for a choosing process in order for us to attribute free will to them. We also don’t attribute free will to watches, as there’s no evidence of generation of action plan.

One may point out that according to the definition, dogs and chess programs have free will in some respect, which seems to conflict with our intuitions. We don’t have to worry here, as human beings have free will on hugely more topics than dogs and chess programs. Even on the same topic, human beings are usually much more creative than dogs, thus exhibit a higher degree of free will. In addition, attributing free will to a chess program does not change the fact that it lacks consciousness, linguistic capabilities, and a lot of values human beings could have.

\subsection{Can Organizations Have Free Will}

Another worry with the definition might be that it seems that we can attribute free will to organizations, such as companies, countries, etc. It’s obvious that in economical activities, a company can generate action plans and decide rationally. Does this justify the attribution of free will to a company? If it’s justified to do so, is free will used in a figurative sense or not? I think it’s justified to attribute free will to a company in economical activities, and the usage of free will here is not in a figurative sense. This view is consistent with the fact that each organization is a legal person in law. Free will is only related to some capacity, nothing more.

\subsection{Is It Compatible with Determinism}

Is it possible?

\subsection{Problem of Responsibility}
