\section{Possible Attacks and Responses}

This section I'll discuss possible attacks on the new conception and respond to them.

\subsection{How does it explain the intuition of free choice}

In the beginning of this paper, we've seen an example where you can decide freely on whether to give the coin in your pocket to the street artist or not. It seems that each choice is at your disposal, it's completely to you to decide. Both for traditional philosophers and ordinary people, this feel of ability to make free choice seems to be essential to the concept of free will, which the new conception failed to capture.

First, I'd like to say, \emph{free will is not hesitation}. When you can't decide between two choices, you don't know what to do next, you feel unease about the uncertainty, you want to get out of the situation. You linger on one choice, while looking at the other. You intend to decide on one, but you hesitate. You swing like a pendulum between the two choices. This is a trouble instead of freedom! It's this misconception of free will that misled previous philosophers for more than thousand years. It's time to give up this myth of free will.

Second, \emph{free will is not randomness}. When facing two equally good choices, we often resort to the strategy of random choice. Resorting to random choice doesn't necessarily mean resort to a true random process in the physical world. It just means to give up our right to decide, and let external forces -- which we can't predict the result -- to decide. Note that the act of resorting to randomness in case of equally good choices is a rational choice, which is captured by the new conception of free will. However, the random decision process itself is unrelated to free will. Free will has nothing to do with randomness.

Third, \emph{free will is reflected in creative generation of action plans and rational decision with respect to an objective}. Let's look at the street artist example again, why you're faced with two choices? That's because you asked yourself:

\begin{quote}
  Should I give the coin in my pocket to the artist?
\end{quote}

Which is the same question as following:

\begin{quote}
  Choose one from following: (1) Give the coin in my pocket to the artist; (2) Don't give the coin in my pocket to the artist.
\end{quote}

So what we're facing is actually a problem! We can come up with a lot of options in order to solve this problem:

\begin{itemize}
\item I give the coin if I don't need the coin today
\item I give the coin if an artist really needs help
\item I give the coin if an artist really performs well
\item I give the coin if I'm happy today
\item I resort to a random choice
\end{itemize}

If we only come up with one option in the concrete situation, then we're done. However,  things seem to be worse, if we come up with more than one choice. How do we choose among the choices?

There's always more than one choice, because of negation!

Also generate objective, the interaction between objective and options.

\subsection{Do Dogs Have Free Will}

The explicit requirement of generating action plans and choosing rationally prohibits us from attributing free will to trees and functional devices. Trees get rid of their leaves in autumn to protect their lives, but there’s no evidence for a choosing process in order for us to attribute free will to them. We also don’t attribute free will to watches, as there’s no evidence of generation of action plan.

One may point out that according to the definition, dogs and chess programs have free will in some respect, which seems to conflict with our intuitions. We don’t have to worry here, as human beings have free will on hugely more topics than dogs and chess programs. Even on the same topic, human beings are usually much more creative than dogs, thus exhibit a higher degree of free will. In addition, attributing free will to a chess program does not change the fact that it lacks consciousness, linguistic capabilities, and a lot of values human beings could have.

\subsection{Can Organizations Have Free Will}

Another worry with the definition might be that it seems that we can attribute free will to organizations, such as companies, countries, etc. It’s obvious that in economical activities, a company can generate action plans and decide rationally. Does this justify the attribution of free will to a company? If it’s justified to do so, is free will used in a figurative sense or not? I think it’s justified to attribute free will to a company in economical activities, and the usage of free will here is not in a figurative sense. This view is consistent with the fact that each organization is a legal person in law. Free will is only related to some capacity, nothing more.

\subsection{Is It Compatible with Determinism}

Is it possible?

\subsection{What's exactly an action plan}

Does a calculator counts? What about a program with only if and else?

\subsection{Problem of Responsibility}
