\section{Empirical Search for A Definition}

In the above, we have examined existing definitions,  but none of them seems satisfactory. Why not try to search for a new definition? Next we are going to inspect a list of objects from simple ones like atoms, to complex creatures like dogs, in order to see what objects we are willing to attribute or reject to attribute free will, and for what reasons.

For objects like water, air, earth, stones, stars, balls, atoms and molecules, we will not attribute free will to any of them, no matter their behavior can be precisely predicted or not. The reason is simple, the movements of these objects is completely determined by their relations to environment.

For bikes, cars and airplanes, we don’t attribute free will either. This is due to the fact that their behavior is controlled by other agents or environment completely.

What about virus, flowers and trees? It seems such things have a will to live and have some desires for water and nutrition(possibly in a figurative sense). It seems they have choices as well, a tree can choose how it grows its roots, a virus can choose which cell to attack, a flower can choose when to bloom. However, we’re reserved to attribute free will to such things, as the evidence is insufficient for us to believe their behaviors are not completely determined by the environment.

What about ants, fishes, mosquitoes and bees? These small animals have intelligent behaviors, and they can adapt their behaviors to the environment in order to attain an objective. Usually we don’t attribute free will to these animals. But why? One possible answer might be that their intelligent behaviors might be just some complex stimulus-response mechanisms shaped partly by inheritance and partly by the environment.

What about intelligent programs, robots? We know that the intelligent program Deep Blue by IBM has been able to defeat world champions of chess and the latest program Watson by IBM  has been able to win television contest. The intelligent programs are being able to weigh different choices and choose the best strategy. Why we don’t attribute free will to these intelligent artifacts? One possible answer might be that their intelligent behaviors are just some complex input-output mechanism.

What about mice, birds, cats, dogs and wolves? What about cows, lions, tigers and elephants? These animals do have desires and beliefs, and we should attribute consciousness to them as well. What they lack seems to be the ability to reflect on their basic desires and beliefs. It’s very likely that these animals are determined by their natural inclinations, thus it makes sense to not attribute free will to them.

But how do we know they can’t reflect on their basic desires and beliefs since we can’t talk with them? Historically, the Europeans also did not attribute free wills to American aboriginals, calling them savages and treat them inhumanly, which turned out to be a prejudice. How do we know it’s not an epistemological prejudice when we refuse to attribute free will to these animals?

What about baby and children? Seldom we attribute free will to a new born baby. But when we can be sure to attribute free will to them? When they don’t obey their parents? When they are being able to speak?

What about drug-addicts? It seems that usually we also don’t attribute free will to them temporarily if they can’t reflect. For one who can reflect their behaviors but can’t mend his way, usually we don’t say that he has no free will, instead we say he lacks will power or his will is weak. This is the same case for kleptomaniacs. Does this mean free will comes in varying degrees?

What about intelligent animals in cartoons? We don’t hesitate to attribute free will to them as they can speak, deliberate, plan, decide and act.

What about aliens? It depends. If they only exhibit patterns of behaviors that can be simply predicted by a set of desires and beliefs, then it’s unlikely that we will attribute free will to them. Otherwise, if they exhibit the ability to reflect on their desires and beliefs and act based on such reflections, we will be ready to attribute free will to them.

From the analysis above, we have seen that the most important reason for us to reject attribute free will to objects are as follows:

\begin{itemize}
\item The object is completely determined by environment
\item The object can’t deliberate and reflect
\item The object is just some complex mechanism
\end{itemize}

The first characterization has the weakness that it’s a negative characterization, because it just says something about what’s not free will. The second characterization depends on concepts like beliefs, desires, actions and reflections, as mind is still a very big mystery, these concepts may invite many disputes and fall into different interpretations. The third characterization is not only negative, but also problematic, as how do we know human beings are not just some very complex mechanisms? This question is itself related to the difficult mind-body problem and is open to many disputes.

So our analysis above is not very helpful. In the next section I will present a new definition of free will.
