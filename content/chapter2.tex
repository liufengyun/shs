\section{Possible Attacks and Responses}

A good boxer should not only be good at attacking, but also defense. In this section I'll discuss possible attacks on the new definition of free will and respond to them.

\subsection{Do Trees, Dogs and Chess Programs Have Free Will}

The explicit requirement of generating action plans and choosing rationally prohibits us from attributing free will to trees. Trees get rid of their leaves in autumn to protect their lives during the winter, but there’s no evidence of generation of action plans in order for us to attribute free will to them. In another word, tree doesn't show creativity at all to deserve the title of free will.

One may point out that according to the definition, dogs and chess programs have free will in some respects, which seems to conflict with our intuitions. We don’t have to worry here, we can still be proud of our free will, as human beings have free will on hugely more topics than dogs and chess programs. Even on the same topic, human beings are usually much more creative than dogs, thus exhibit a higher degree of free will. In addition, attributing free will to a chess program does not change the fact that it lacks consciousness, linguistic capabilities, and a lot of values human beings could have.

\subsection{Do Organizations Have Free Will}

Another worry with the definition might be that it seems that we can attribute free will to organizations, such as companies, political parties, etc. It’s obvious that in economical activities, a company can generate action plans creatively and decide rationally. Does this justify the attribution of free will to a company? If it’s justified to do so, is free will used in a figurative sense or not? I think it’s justified to attribute free will to a company in economical activities, and the usage of free will here is not in a figurative sense. This view is consistent with the fact that organizations are valid subjects of crime in economical laws, but not in traffic laws. Free will is only related to some capacity, nothing more.

\subsection{How Many Action Plans are Required}

If we play arithmetic game with a child, and the child manages to answer all questions related to simple addition and subtraction questions, we would praise the child -- no doubt we would attribute free will to the child. But what if we play the arithmetic game with a calculator? Should we attribute free will to a calculator?

And we can imagine an even simpler game as follows:

\begin{quote}
  Your task is to guess what will be the output of the program, and make your guess known to the program. If you guess 1, the program would output 0. If you guess 0, the progress would output 1.
\end{quote}

This is a game that you are destined to loose. It seems silly, but it's still OK to call it a game. The question is, does this game qualify the name of free will?

The similarity between the calculator and the game is that they go directly to the optimal solution. But that's also the same way how masters in a field work. They are so experienced in their skills that they seem to work out of instinct. An awkward speaker may need to deliberate on the wording, but an orator speaks just like the flow of water. An experienced student in calculus can directly produce the integral of a function, while an inexperienced student may have to try different methods.

So how many action plans should be generated in order to qualify free will? Can we attribute free will to an agent with just one action plan? We can attribute free will to masters in a field even there's only one action plan, but I hesitate to attribute free will to the calculator and simple game above, as they are not agents, they have no values. This crucial property differentiates them from skilled masters in a field.

\subsection{Is It Compatible with Determinism}

To answer this question, let's recall the formulation of the incompatibility between determinism and free will:

\begin{quote}
 P1 If determinism is true, then every human action is causally necessitated \\
 P2 If every action is causally necessitated, no one could have acted otherwise \\
 P3 One only has free will if one could have acted otherwise \\
 P4 Determinism is true \\
 C No one has free will
\end{quote}

As we have a new definition of free will, I would change P3 to P3* as follows:

\begin{quote}
  P3* One only has free will if one showed creativity in generating choices
\end{quote}

After this change, it seems free will and determinism are compatible, as creativity doesn't depend on magical powers. In the real world, we've seen a lot of programs and robots are capable to learn and show creativity in solving problems.

Some might point out that is it possible our creativity is determined? I agree it's a possibility, but I doubt determinism is an accurate, complete and final image of the complex world. The new definition of free will is neutral to different metaphysical stances of the world. \emph{What's free will} and \emph{how is free will possible} are two different questions. The answer to the first question doesn't depend on the second. However, the answer to the second question depends on the answer to the first one.

\subsection{Is Responsibility Still Possible}

Generally speaking, there are two broad philosophical interpretations of the concept responsibility, namely \emph{merit-based view} and \emph{consequentialist view}\cite{sep-moral-responsibility}.

According to the \emph{merit-based view}, praise or blame would be an appropriate reaction toward the agent if and only if he merits such a reaction. For blame or praise to be appropriate, there must be actual alternatives open to the agent and the agent should have acted freely, as the slogan goes \emph{no responsibility without freedom}.

However, according to \emph{consequentialist view}, praise or blame would be appropriate if and only if a reaction of this sort would likely lead to a better change, e.g. the agent improves his behavior. For most compatibilists, blaming and praising are just means of social regulation.

Regarding the new conception of free will, if one does something creatively, then he deserves the reward. Some may disagree that the creativity is historically determined. I'd like to say that the object of the reward is a person who has a body and would one day age and die. His creativity might be determined, but it takes part of his life to acquire and maintain the creativity. So in this case, the \emph{merit-based view} applies. With this line of argument, I also think that trained animals like police dogs also deserve reward in their creative performance, even it's insane to do so from the consequentialist view.

However, when a person lacks the creativity required to performance well, it seems the merit-based view doesn't apply, because this situation is determined. In this case, \emph{consequentialist view} applies -- it makes sense to reward, punish or show tolerance as long as the reaction helps the person to improve his creativity.

%% \subsection{How Does Creativity Relate to Moral Responsibility}

%% It seems that when one helps the other or one traps the other, what's involved is the person's will or values. How does creativity relate to moral responsibility?

%% I'd like to say that there are a lot of reasons for doing something good to others -- religious reasons, ethical thinking, selfish calculations, etc. Such behaviors often exhibit creative understandings of fortune, happiness, self, others and the world.

%% For people who do bad deeds, it's usually the case that they lack the creativity to deal with their corporal or psychological propensities.
