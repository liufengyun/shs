\documentclass[xcolor=dvipsnames]{beamer}

\usepackage{amsmath, amssymb, graphicx}
\usepackage[english]{babel}
\usepackage{times}
\usepackage[utf8]{inputenc}
\usepackage[T1]{fontenc}
\usepackage{listings}
\usepackage{hyperref}
\usepackage[norelsize,ruled,vlined]{algorithm2e}
\usepackage{color}
\usepackage{hyperref}
\usepackage{booktabs}
\usepackage{tikz}
\usetikzlibrary{matrix}
\usetikzlibrary{arrows}
\usetikzlibrary{positioning}
\usetikzlibrary{shapes.multipart}

\theoremstyle{definition}
\newtheorem{proposition}{Proposition}

\mode<presentation>
\usecolortheme{fly}

\setbeamercolor{alerted text}{fg=purple}

\setbeamertemplate{footline}{%
  \usebeamercolor[fg]{navigation symbols}%
  \usebeamerfont{footline}
  \parbox{\linewidth}{
    \vspace*{-8pt}
    \hspace{5pt}\insertpagenumber/\inserttotalframenumber
    %% \insertshortauthor\hspace{5pt}\insertshortinstitute
    %% \hfill \insertsection
    %% \hfill\usebeamertemplate***{navigation symbols}
    %% \hfill
  }
}
%% \setbeamertemplate{navigation symbols}{}


\title[Free Will Is Creative]{Free Will Is Creative\\
  {\small Towards A New Conception of Free Will}
}
\author{Fengyun Liu}
\institute[EPFL]{EPFL}
\date{\today}

%% set code styles

\definecolor{dkgreen}{rgb}{0,0.6,0}
\definecolor{gray}{rgb}{0.5,0.5,0.5}
\definecolor{mauve}{rgb}{0.58,0,0.82}

\lstset{
  language=scala,
  aboveskip=3mm,
  belowskip=3mm,
  showstringspaces=false,
  columns=flexible,
  basicstyle={\ttfamily\Large},
  numbers=none,
  numberstyle=\tiny\color{gray},
  keywordstyle=\color{blue},
  commentstyle=\color{dkgreen},
  stringstyle=\color{mauve},
  breaklines=true,
  breakatwhitespace=true,
  tabsize=3,
}


% Delete this, if you do not want the table of contents to pop up at
% the beginning of each subsection:
\AtBeginSection[]
{\begin{frame}<beamer>{Overview}
        \tableofcontents[
            sections={1-6},
            currentsection,
            currentsubsection,
            hideothersubsections,
            sectionstyle=show/shaded,
        subsectionstyle=show/shaded/hide]
    \end{frame}
}

\begin{document}


%%%%%%%%%%%%%%%%%%%%%%%%%%%%%%%%%%%%%%%%%%%%%%%%%%%%%%%%%%%%%%%
% 0. Titlepage
%%%%%%%%%%%%%%%%%%%%%%%%%%%%%%%%%%%%%%%%%%%%%%%%%%%%%%%%%%%%%%%
{
\setbeamertemplate{footline}{}
\begin{frame}[noframenumbering]
    \titlepage{}
\end{frame}


%% \begin{frame}[noframenumbering]{Today's agenda}
%% \tableofcontents[hideallsubsections,
%%     sections={1-6}
%% ]
%% \end{frame}
}

%%%%%%%%%%%%%%%%%%%%%%%%%%%%%%%%%%%%%%%%%%%%%%%%%%%%%%%%%%%%%%%
% 1. Background
%%%%%%%%%%%%%%%%%%%%%%%%%%%%%%%%%%%%%%%%%%%%%%%%%%%%%%%%%%%%%%%
% subsection Motivation (end)
\section{Traditional View of Free Will} % (fold)
\label{sec:Background}

\begin{frame}[fragile]
  \frametitle{Free Will as the Capacity to Choose}
  Traditional philosophers characterize free will as the \alert{capacity to choose}.

  \begin{itemize}
  \item Libertarians: \emph{The ability to do otherwise}
  \item Compatibilists: \emph{The ability to align desires and actions}
  \end{itemize}
\end{frame}

\begin{frame}[fragile]
  \frametitle{The Magic Power of Choice}
  An agent has free will can do either A or B, solely based on his/her choice.

  \begin{figure}
    \centering
    \includegraphics[width=0.8\textwidth]{images/woods.jpg}\\
    \emph{Two Roads Diverged In a Yellow Wood}
  \end{figure}
\end{frame}

\begin{frame}[fragile]
  \frametitle{The Magic Power of Choice}
  Once owned this magic power, it can be used \alert{anywhere}, \alert{anytime} and on \alert{any matters}.
  \begin{figure}
    \centering
    \includegraphics[width=0.8\textwidth]{images/magic.jpg}\\
  \end{figure}
\end{frame}

\begin{frame}[fragile]
  \frametitle{Two Tasks of Presentation}
  I dub the view which regards \emph{free will as the magic power of choice} as the \alert{myth of choice}.\\[1cm]

  Tasks of this presentation:
  \begin{itemize}
  \item Break the \emph{myth of choice}
  \item Show that \emph{free will is creative}
  \end{itemize}
\end{frame}

%%%%%%%%%%%%%%%%%%%%%%%%%%%%%%%%%%%%%%%%%%%%%%%%%%%%%%%%%%%%%%%
% 2. The Myth of Choice
%%%%%%%%%%%%%%%%%%%%%%%%%%%%%%%%%%%%%%%%%%%%%%%%%%%%%%%%%%%%%%%
\section{The Myth of Choice} % (fold)

\begin{frame}[fragile]
  \frametitle{Free Will Is Not Irrationality}
  Rationality means you should choose the better one!

  \begin{figure}
    \centering
    \includegraphics[width=0.8\textwidth]{images/rationality.jpg}\\
    \emph{Only an insane deliberately makes an suboptimal choice!}
  \end{figure}
\end{frame}

\begin{frame}[fragile]
  \frametitle{Rationality Is Not A Mysterious Capacity}

  \begin{lstlisting}[language=Scala]
    def decide(opt1, opt2) = {
      if (opt1 betterThan opt2)
          choose(opt1)
      esle if (opt2 betterThan opt1)
          choose(opt2)
      else
          ...
    }
  \end{lstlisting}
\end{frame}

\begin{frame}[fragile]
  \frametitle{Free Will Is Not Hesitation}
  If A and B are equally good, then you're in trouble!

  \begin{figure}
    \centering
    \includegraphics[width=0.8\textwidth]{images/ass.jpg}\\
    \emph{Buridan's ass}
  \end{figure}
\end{frame}

\begin{frame}[fragile]
  \frametitle{Hesitation Is Not A Mysterious Capacity}

  \begin{lstlisting}[language=Scala]
    def decide(opt1, opt2) = {
      if (opt1 betterThan opt2)
          choose(opt1)
      else if (opt2 betterThan opt1)
          choose(opt2)
      else
          decide(opt2, opt1)
    }
  \end{lstlisting}
\end{frame}

\begin{frame}[fragile]
  \frametitle{Free Will Is Not Randomness}
  If you resort to randomness, you are giving up free will!

  \begin{figure}
    \centering
    \includegraphics[width=0.8\textwidth]{images/dice.jpg}\\
    \emph{You are determined by external forces that you can't predict}
  \end{figure}
\end{frame}

\begin{frame}[fragile]
  \frametitle{Randomness Is Not A Mysterious Capacity}

  \begin{lstlisting}[language=Scala]
  def decide(opt1, opt2) = {
    if (opt1 betterThan opt2)
       choose(opt1)
    else if (opt2 betterThan opt1)
       choose(opt2)
    else
       choose(random(opt1, opt2))
  }
  \end{lstlisting}
\end{frame}

\begin{frame}[fragile]
  \frametitle{Free Will Is Not the Capacity to Decide}

  The \alert{myth of choice} bankrupts:
  \begin{itemize}
  \item The capacity to decide can't differ from the program
  \item There's nothing special about the capacity to decide
  \item The capacity to decide can't honor the name free will
  \end{itemize}
\end{frame}


%%%%%%%%%%%%%%%%%%%%%%%%%%%%%%%%%%%%%%%%%%%%%%%%%%%%%%%%%%%%%%%
% 3. Free Will is Creative
%%%%%%%%%%%%%%%%%%%%%%%%%%%%%%%%%%%%%%%%%%%%%%%%%%%%%%%%%%%%%%%
\section{Free Will is Creative} % (fold)
\label{sec:creative}

%%%%%%%%%%%%%%%%%%%%%%%%%%%%%%%%%%%%%%%%%%%%%%%%%%%%%%%%%%%%%%%
% 4. Conclustion
%%%%%%%%%%%%%%%%%%%%%%%%%%%%%%%%%%%%%%%%%%%%%%%%%%%%%%%%%%%%%%%
\section{Conclusion} % (fold)
\label{sec:conclusion}

\end{document}
